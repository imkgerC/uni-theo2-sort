\chapter{Programm- bzw. Quellcode}
\section{Quellcode}
Der komplette Quellcode inklusive Graph-Generierung wurde in Rust geschrieben. Ausgeführt der Code über das Tool \enquote{cargo}. Die Konfiguration der Messung erfolgt über den Quellcode.

Da gefordert wurde, dass der komplette Quellcode im Dokument enthalten ist, folgt eine Sektion mit demselben. Alternativ kann das Projekt auch angenehm auf GitHub unter \footnote{https://github.com/imkgerC/uni-theo2-sort} eingesehen werden. Die Dokumentation und Erklärung des Codes ist über \enquote{Doc Comments} realisiert, kann also im Code durch Kommentare gelesen werden.
\section{Vollständiger Quellcode}
\lstinputlisting[label=code:main, caption=main.rs, language=Rust, style=colouredRust]{../src/main.rs}
